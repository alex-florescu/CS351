

\subsection*{Mandatory Objectives}
The core objective of the project is to design a series of mono audio effects (which are commonly used by guitar players) and implement them in SystemVerilog to produce a functional FPGA audio processor. This would consist of:
\begin{itemize}
    \item Bypass option: permit the sound to pass through the system with no alterations.
    \item Delay effect: create an effect where a limited number of delayed copies of the sound are added on top of the input sound.
    \item Distortion effect: create an effect where the input sound signal is amplified and clipped, producing a \textit{fuzzy} tone.
    \item Effect selector: implement a method to select one (or none) of the effects offered by the audio processor.
\end{itemize}

\subsection*{Optional Objectives}
Depending on the progress and pace of the project, other effects and features can be added to the audio processor, such as:
\begin{itemize}
    \item Reverb effect: adding multiple slightly delayed and lowered copies of the sound on top of the input sound.
    \item Effects pipelining: allow multiple effects to be selected at the same time, in a desired order.
\end{itemize}

\subsection*{Potential Objectives}
If the prior objectives are accomplished far before the project deadline, some other features can be added to the audio processor, such as:
\begin{itemize}
    \item Noise cancellation effect: removing an unwanted sound of a certain frequency from the input sound.
    \item Tuner: LEDs on the board can be programmed to provide information useful for tuning the guitar.
\end{itemize}