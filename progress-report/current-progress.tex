Talk about research on Audio ports and their configuration based on the Zybo reference manual

maybe talk about the constraints file?
% (https://digilent.com/reference/programmable-logic/guides/vivado-xdc-file#:~:text=This%20file%20contains%20the%20constraints,bank%20voltages%2C%20for%20some%20examples.)


Talk about I2S

Talk about Zybo reference manual configurations and what each port means

Talk about important aspects of DMA project?


\subsection*{Technical Progress}


The first objective in the development of the project is to design a an FPGA module that directly connects the audio inputs to the outputs.

To create this, an analysis was carried out on the functionality and format of the audio input and output ports of the target device (Digilent Zybo Z7).

The Zybo Z7 board contains three analogue 3.5mm standard jack ports. The analogue signals received through these connections are converted using the board's integrated ADC (analogue-to-digital converter). Therefore, from the perspective of the FPGA designer, the inputs can only be seen as 1-bit digital signals.

Based on the reference manual from the board manufacturer, an investigation was conducted on how these signals must be interpreted and configured.

\subsubsection{I2S Protocol}
The audio data is transferred using the I2S (Inter-IC Sound) protocol, a serial link designed especially for transmitting digital audio data.
% (https://web.archive.org/web/20070102004400/http://www.nxp.com/acrobat_download/various/I2SBUS.pdf)


Add a figure about I2S signals


As shown in the figure, the bus has three lines (each of them of size 1-bit):

\begin{enumerate}
    \item Continuous Serial Clock (SCK)
    \begin{itemize}
        \item A clock generated by the controller device, that determines the data transmission rate.
    \end{itemize}
    
    \item Serial Data (SD)
    \begin{itemize}
        \item Audio data transmitted in two's complement. 
        \item The MSB is transmitted first because the transmitter and receiver may have different word lengths.
    \end{itemize}

    \item Word select (WS)
    \begin{itemize}
        \item The word select line indicates the channel being transmitted:
        \begin{itemize}
            \item WS = 0; channel 1 (left)
            \item WS = 1; channel 2 (right)
        \end{itemize}
        \item The WS signal changes one clock cycle after the MSB is transmitted, which allows a clock cycle for other necessary calculations
        \item The WS is also generated by the controller device
    \end{itemize}

\end{enumerate}
% (https://www.nxp.com/docs/en/user-manual/UM11732.pdf)






>> Talk about the ports on the Zybo: modify the table below, use the data from the Zybo.md

\begin{table}[h]
    \centering
    \begin{tabular}{|c|p{8cm}|}
        \hline
        \textbf{Signal} & \textbf{Description} \\
        \hline
        MCLK & Master/slave configuration (decides direction of BCLK, PBLRC, and RECLRC) \\
        \hline
        BCLK & Clock on which data is sent bit by bit \\
        \hline
        PBLRC/RECLRC & Word select (Left/Right) signals for Output and Input ports respectively \\
        \hline
        PBDAT/RECDAT & 1-bit data bus for Output and Input channels respectively \\
        \hline
        MUTE & Active low mute output. Set to 1 to enable audio output \\
        \hline
        SCL & I2C clock \\
        \hline
        SDA & I2C data \\
        \hline
    \end{tabular}
    \caption{Signal Descriptions}
    \label{tab:signals}
\end{table}

>> talk about i2s
>> add figures and reference them for i2s transmissions

As the data in I2S format is stored in 1 bit, therefore we cannot directly perform any calculation on our values. The next objective is to use the information we know about the I2S configuration to store the data into registers.

>> talk about the dma audio project

Another crucial feature of Digilent's DMA project that we need to recreate in our audio processor is the usage of FIFO modules on both input and output registers. Without this we would not be able to perform operations on data from different points in time.



\subsection{Upcoming Work}

Talk about future work: the effects

Talk about the delay effect

Maybe you can talk about the signal processing module and how that has proved helpful