According to Xilinx, FPGAs (Field Programmable Gate Arrays)\cite{whatisanfpga} are semiconductor devices that can be reprogrammed to desired application and functionality after manufacturing. They consist of a matrix of configurable logic blocks connected via programmable interconnects.

The primary aim of the project is to use an FPGA board to recreate a series of audio effects commonly used for guitars, such as \textit{delay}, \textit{distortion} and \textit{reverb}, along with a \textit{mechanism for selection} between them. Additional effects might be implemented during later stages of the project, depending on available time. 

\textit{Audio effects} can be defined as ``analogue or digital devices that purposely change the sound of a music instrument or other audio source using different techniques or devices.''\cite{audio-effects} Audio effects for guitars are usually generated either by using physical guitar pedals or via software. Implementing such effects on an FPGA will provide the advantages of an \textit{upgradable} and \textit{highly customisable} audio processor while maintaining the convenience of a \textit{compact device}.